\documentclass{beamer}
\usepackage[utf8]{inputenc}
\usepackage[T1]{fontenc}
\usepackage[spanish]{babel}
\usepackage{graphicx}
\usepackage{booktabs}

% Theme settings for PDF compatibility (Pandoc maps these to structure)
\usetheme{Madrid}
\usecolortheme{beaver}

\title{FOLLETO ITINERARIO FAMILIAR – VALENCIA}
\subtitle{19 y 20 de DICIEMBRE ✨}
\author{Plan de Viaje Familiar}
\date{Diciembre 2025}

\begin{document}

\frame{\titlepage}

\begin{frame}
\frametitle{¡Bienvenidos a España! 🇪🇸}
\begin{center}
\Large Un viaje emocionante para disfrutar en familia
\end{center}
\vspace{1em}
Este itinerario está diseñado para aprovechar al máximo dos días mágicos en Valencia y sus alrededores, combinando cultura, gastronomía y relax antes del viaje a Holanda.
\end{frame}

\section{Resumen del Viaje}

\begin{frame}
\frametitle{Resumen del Itinerario}
\begin{itemize}
    \item \textbf{Día 1 (Jueves 19):} Inmersión en el corazón de Valencia, tapeo histórico y cena modernista.
    \item \textbf{Día 2 (Viernes 20):} Relax en Onda y Castellón, brisa marina en Valencia y noche mágica de música.
\end{itemize}
\end{frame}

\section{DÍA 1: Valencia Histórica}

\begin{frame}
\frametitle{🌟 DÍA 1 — JUEVES 19 DE DICIEMBRE}
\begin{center}
\Huge Descubriendo Valencia
\end{center}
\end{frame}

\begin{frame}
\frametitle{🕚 11:00 – 12:30 | Joyas del Centro Histórico (I)}
\textbf{Plaza del Ayuntamiento 🏛️}
\begin{itemize}
    \item El corazón administrativo y festivo de la ciudad.
    \item Edificios majestuosos y ambiente vibrante.
\end{itemize}

\vspace{1em}

\textbf{Plaza Redonda 🎨}
\begin{itemize}
    \item Un rincón único por su forma circular.
    \item Comercios tradicionales y artesanía local.
\end{itemize}
\end{frame}

\begin{frame}
\frametitle{🕚 11:00 – 12:30 | Joyas del Centro Histórico (II)}
\textbf{Mercado Central 🍊}
\begin{itemize}
    \item Una catedral de la gastronomía.
    \item Arquitectura modernista espectacular llena de luz y color.
\end{itemize}

\vspace{1em}

\textbf{Lonja de la Seda 🏰}
\begin{itemize}
    \item Patrimonio de la Humanidad por la UNESCO.
    \item Obra maestra del gótico civil valenciano.
\end{itemize}
\end{frame}

\begin{frame}
\frametitle{🍽️ 12:30 – 14:00 | La Experiencia del Tapeo}
\textit{Paradas obligatorias para un tapeo tradicional:}

\begin{itemize}
    \item \textbf{Casa Baldo:} Tradición en cada bocado.
    \item \textbf{Boatella Tapas:} Famoso por sus productos frescos frente al mercado.
    \item \textbf{Taberna El Clavo:} Autenticidad y sabor local.
    \item \textbf{La Estrecha:} Un lugar con historia y encanto (la fachada más estrecha).
\end{itemize}
\end{frame}

\begin{frame}
\frametitle{🌳 15:00 – 17:00 | Pulmón Verde: Jardín del Turia}
Un paseo para desconectar en el parque urbano más largo de Europa.

\begin{itemize}
    \item \textbf{Actividades:} Caminar tranquilamente bajo los árboles.
    \item \textbf{Recuerdos:} El escenario perfecto para fotos familiares.
    \item \textbf{Historia:} Antiguo cauce del río convertido en jardín.
\end{itemize}
\end{frame}

\begin{frame}
\frametitle{🏙️ 17:00 – 19:30 | Tarde de Compras y Paseo}
Rumbo hacia la zona del Ensanche y Mercado de Colón.

\begin{itemize}
    \item Disfrutar de la arquitectura señorial de las calles.
    \item Tiempo libre para entrar a tiendas y boutiques.
    \item Parada técnica para un café o merienda relajada.
\end{itemize}
\end{frame}

\begin{frame}
\frametitle{🌙 19:30 – 21:30 | Cena en Mercado Colón}
Una joya del modernismo valenciano con ambiente navideño.

\textbf{Opciones Gastronómicas:}
\begin{itemize}
    \item \textbf{Ma Khin Café:} Fusión y sabor en un entorno elegante.
    \item \textbf{Habitual:} La propuesta mediterránea de Ricard Camarena.
    \item \textbf{Puro Osmos:} Calidad y producto.
\end{itemize}
\end{frame}

\section{DÍA 2: Preparativos y Cultura}

\begin{frame}
\frametitle{🌟 DÍA 2 — VIERNES 20 DE DICIEMBRE}
\begin{center}
\Huge Preparativos y Cultura
\end{center}
\end{frame}

\begin{frame}
\frametitle{☀️ 08:00 – 11:00 | Mañana en Onda}
\textbf{Objetivo: Todo listo para Holanda 🧳}

\begin{itemize}
    \item Organización de maletas y equipaje.
    \item Revisión de documentos (DNI/Pasaportes) y billetes.
    \item Cargar todos los dispositivos móviles.
    \item Dejar lista la ropa para el viaje del día siguiente.
\end{itemize}
\end{frame}

\begin{frame}
\frametitle{🏙️ 11:00 – 13:00 | Relax en Castellón}
Un paseo tranquilo antes de volver a la capital.

\begin{itemize}
    \item \textbf{Plaza Mayor:} El centro neurálgico de la ciudad 🏛️.
    \item \textbf{Concatedral de Santa María:} Arquitectura gótica reconstruida ⛪.
    \item \textbf{Parque Ribalta:} Un paseo entre naturaleza 🌳.
    \item \textbf{Pausa dulce:} Chocolate con churros tradicional ☕.
\end{itemize}
\end{frame}

\begin{frame}
\frametitle{🌊 15:30 – 17:00 | Brisa del Mediterráneo}
Tras el traslado a Valencia (14:30), disfrutaremos del mar.

\textbf{La Marina o Playa de la Malvarrosa:}
\begin{itemize}
    \item Paseo marítimo con vistas al Mediterráneo.
    \item Sesión de fotos junto al mar.
    \item \textbf{La Más Bonita:} Parada recomendada para tomar algo dulce en un ambiente ibicenco.
\end{itemize}
\textit{Alternativa: Volver al Jardín del Turia si se prefiere centro.}
\end{frame}

\begin{frame}
\frametitle{🎻 19:00 | Candlelight: Una Noche Mágica}
\textbf{Lugar: Ateneo Mercantil}

\begin{itemize}
    \item \textbf{Horario:} Llegada recomendada 18:15 – 18:30.
    \item \textbf{Experiencia:} Concierto a la luz de las velas.
    \item \textbf{Ambiente:} Íntimo, relajado y familiar, perfecto para recordar.
\end{itemize}
\end{frame}

\begin{frame}
\frametitle{🍽️ 20:15 – 21:45 | Cena de Despedida}
\textbf{Lugar: Begîn – Cortes Valencianas}

Un restaurante con decoración única y platos para todos:
\begin{itemize}
    \item Bao de shiitake 🍄
    \item Burgers plant-based 🍔
    \item Nachos para compartir 🧀
    \item Cheesecake vegano 🍰
\end{itemize}
\end{frame}

\begin{frame}
\frametitle{🚗 22:00 | Regreso y Descanso}
Vuelta a Onda para descansar.

\begin{center}
Todo debe quedar listo para la aventura en Holanda al día siguiente.
\end{center}
\end{frame}

\section{Consejos Finales}

\begin{frame}
\frametitle{✅ Tips para el Viaje}
\begin{enumerate}
    \item \textbf{Ropa:} Capas cómodas y abrigo, ¡es diciembre!
    \item \textbf{Calzado:} Zapatillas cómodas, caminaremos bastante.
    \item \textbf{Tecnología:} Powerbank para los móviles (muchas fotos).
    \item \textbf{Dinero:} Algo de efectivo para pequeños gastos en mercadillos.
    \item \textbf{Actitud:} ¡Disfrutar de cada momento en familia!
\end{enumerate}
\end{frame}

\begin{frame}
\frametitle{¡Buen Viaje!}
\begin{center}
\Huge ¡A disfrutar de Valencia y Holanda! ✈️
\end{center}
\end{frame}

\end{document}
